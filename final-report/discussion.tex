\section{Discussion (\textasciitilde 1 column)}
\label{sec:discussion}

Guide Dog works well within the limits of what its prototype components were
designed for. However, as a complete system, Guide Dog falls short of the 
vision its authors had originally intended for it. It is only able to track
brightly colored objects, it requires that the ground plane be within view
at all times, and it cannot handle the case where the destination is outside
the viewing range of the camera. 

The most obvious next step for the destination detection component is to
replace it with a more sophisticated, machine learning based algorithm that
is able to learn and detect common household objects with high precision.
Such an algorithm would probably be based on feature learning, such as
previous object-detection work~\cite{lai_icra12}.
This would enable the system to detect and guide the user to real-world
objects of practical interest to the user, greatly increasing the
utility of the system.

An improvement of even greater scope, which would benefit the system as
a whole, would be the addition of a 3D-mapping component and a system
for localizing the cameras position within a pre-existing 3D map~\cite{Du:2011:IMI:2030112.2030123}. This
would allow the system to be placed into a previously-mapped space,
determine its location, and guide the user to a distant point well
outside the line of sight of the camera. A reasonably sophisticated 
localization component would remove the need to see the floor at all
times, enabling more sophisticated obstacle detection, and it would 
allow shortest paths to be computed through a 3D map, enabling audio
guidance with more detailed cues for guiding the user along the path
(e.g. synthetic speech).

The open source support for RGB-D image processing was impressive. Many of the
``hard'' parts of the project were already implemented. For example, PCL's
planar segmentation worked perfectly to detect the ground plane for obstacle
detection. OpenCV's blob detection was simple to set up and use and was taken
advantage of in both destination and obstacle detection.

In the future, in would be beneficial to have a fully integrated system
completed earlier. With a fully integrated system it would be easier to test the
audio interface in a real world environment, which would allow for better
iteration and improvement of the audio system.
