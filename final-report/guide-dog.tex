% THIS IS SIGPROC-SP.TEX - VERSION 3.1
% WORKS WITH V3.2SP OF ACM_PROC_ARTICLE-SP.CLS
% APRIL 2009
%
% It is an example file showing how to use the 'acm_proc_article-sp.cls' V3.2SP
% LaTeX2e document class file for Conference Proceedings submissions.
% ----------------------------------------------------------------------------------------------------------------
% This .tex file (and associated .cls V3.2SP) *DOES NOT* produce:
%       1) The Permission Statement
%       2) The Conference (location) Info information
%       3) The Copyright Line with ACM data
%       4) Page numbering
% ---------------------------------------------------------------------------------------------------------------
% It is an example which *does* use the .bib file (from which the .bbl file
% is produced).
% REMEMBER HOWEVER: After having produced the .bbl file,
% and prior to final submission,
% you need to 'insert'  your .bbl file into your source .tex file so as to provide
% ONE 'self-contained' source file.
%
% Questions regarding SIGS should be sent to
% Adrienne Griscti ---> griscti@acm.org
%
% Questions/suggestions regarding the guidelines, .tex and .cls files, etc. to
% Gerald Murray ---> murray@hq.acm.org
%
% For tracking purposes - this is V3.1SP - APRIL 2009

\documentclass{acm_proc_article-sp}
\usepackage{hyperref}

\begin{document}

\conferenceinfo{CSE 481 O}{Final Report}

\title{Guide Dog: an audio guidance tool}

%
% You need the command \numberofauthors to handle the 'placement
% and alignment' of the authors beneath the title.
%
% For aesthetic reasons, we recommend 'three authors at a time'
% i.e. three 'name/affiliation blocks' be placed beneath the title.
%
% NOTE: You are NOT restricted in how many 'rows' of
% "name/affiliations" may appear. We just ask that you restrict
% the number of 'columns' to three.
%
% Because of the available 'opening page real-estate'
% we ask you to refrain from putting more than six authors
% (two rows with three columns) beneath the article title.
% More than six makes the first-page appear very cluttered indeed.
%
% Use the \alignauthor commands to handle the names
% and affiliations for an 'aesthetic maximum' of six authors.
% Add names, affiliations, addresses for
% the seventh etc. author(s) as the argument for the
% \additionalauthors command.
% These 'additional authors' will be output/set for you
% without further effort on your part as the last section in
% the body of your article BEFORE References or any Appendices.

\numberofauthors{3} %  in this sample file, there are a *total*
% of EIGHT authors. SIX appear on the 'first-page' (for formatting
% reasons) and the remaining two appear in the \additionalauthors section.
%
%\author{
% You can go ahead and credit any number of authors here,
% e.g. one 'row of three' or two rows (consisting of one row of three
% and a second row of one, two or three).
%
% The command \alignauthor (no curly braces needed) should
% precede each author name, affiliation/snail-mail address and
% e-mail address. Additionally, tag each line of
% affiliation/address with \affaddr, and tag the
% e-mail address with \email.
%

\author{
Robart Bart
\qquad
Minh-Quan Nguyen
\qquad
Eric Spishak\\
%
{\normalsize
University of Washington}\\
\url{{rbart,nguyenmq,espishak}@cs.washington.edu}
}

\maketitle
\begin{abstract}
RGB-D cameras, like the Microsoft Kinect, are gaining popularity. Currently, the
technology is used mostly in games. However, the technology is relatively young
and the applications for it are still unknown.

Guide Dog is an experimental application of an RGB-D camera, that
uses audio to direct a user to a destination, while avoiding obstacles. The
system relies on an RGB-D camera mounted on the user to detect the features of
the environment, which are then communicated by audio to the user. It uses
various audio tones to convey the directions and distances of the destination
and obstacles. Guide Dog could be used by a blind person to navigate through a
room or to help a user find his or her keys in the dark.

This paper describes the authors' experiences designing, building and improving
this system. Included is a detailed description of the technologies used and the
paths the authors took to get a final, working product.

Guide Dog's implementation is available online for anyone to try and improve
upon.
\end{abstract}

\keywords{Keywords of your choosing.} % NOT required for Proceedings

\section{Introduction}
\label{sec:intro}

RGB-D cameras like the Microsoft Kinect\cite{kinect-website} have recently
become popular because of their use in video game consoles. These cameras are
unique because not only to do they record color information, but they also record
depth information. This allows the gaming system to do a much better job in
figuring out what is happening in an environment. For example, it is much easier
to detect a person with and RGB-D camera than a color camera. With just a color
camera, complex and expensive computer vision techniques are required to detect
a person. However, this is simplified with an RGB-D camera because the person
will stand out from the background in the depth image.

Currently, the main use of RGB-D cameras is tracking people and using their
movements as inputs in a game. The RGB-D camera is generally placed facing the
user, in front of his or her TV. It captures the environment as the user moves
and passes this information to the video game system. The video game system then
processes this information. It generally extracts the location and movement of a
person in order to control the game. This is an exciting, but limited use of
this technology. Since the technology is young, not many other applications of
the technology have been explored.

Guide Dog explores a new application of RGB-D camera technology. Instead of
being used in a game, Guide Dog is a system that uses audio cues to direct a
user to a destination, while helping the user avoid obstacles. Guide Dog is a
unique application for an RGB-D camera since it is mounted on the user, rather
than facing the user, like general RGB-D camera applications. Guide Dog sees the
environment from the perspective of the user and is able to detect both the
destinations and any obstacles. Guide Dog communicates these locations to the
user by using two different audio tones: one for the destination and one for the
obstacles. The destination tone varies in pitch and direction. The pitch of the
tone indicates the distance from the destination and the direction indicates the
direction to the destination. The obstacle tone only starts when an obstacle is
within a few feet of the user, as only the close obstacles matter and too many
obstacles would overwhelm the user. The obstacle tone also uses direction to
indicate the direction of the obstacles.

This paper describes in detail the Guide Dog system as well as the authors'
experiences designing, building and improving Guide Dog. This paper describes
the various different approaches the authors experimented with when building
Guide Dog. It gives details about what worked, what didn't work, and why. It
gives details of how the individual components work and how they fit together
to form the Guide Dog system.

The paper is organized as follows. Section~\ref{sec:related} describes related
word. Section~\ref{sec:technical} describes the technical details of Guide Dog.
Section~\ref{sec:eval} evaluates the performance and success of Guide Dog.
Section~\ref{sec:discussion} discusses Guide Dog and
Section~\ref{sec:conclusion} concludes.


\section{Related Work}
\label{sec:related}

% Describe other research projects, commercial products, patents, etc that are
% related to your project and how they differ from your work. Also, if your
% work is based on previous techniques, describe them here. Should be about 1
% column.

The System for Wearable Audio Navigation (SWAN)~\cite{Walker04auditorynavigation} is a project
from the Sonification Lab at the Georgia Institute of Technology which aims to
provide a system to help users find their way through an environment. This
closely matches the goals the authors intended for Guide Dog. SWAN features an
audio interface that uses beacons to guide a user to waypoints along a path
determined by the system and object sounds to signal the location of different
objects in the user's environment. Guide Dog borrows these two concepts in the
implementation of its own audio interface. See section
\ref{sec:technical-audio-op} for further details on Guide Dog's audio system.
Another major part of SWAN is the use of virtual reality to aid the development
of its audio interface. The lab uses their virtual reality test environment to
try out new features for SWAN. While developing Guide Dog, the authors emulated
this strategy so that the audio interface could be developed in parallel with
the other portions of Guide Dog.

Where Guide Dog differs from SWAN is how it views and processes the outside
world. SWAN consists of a lightweight computer and several sensors such as GPS,
inertial sensors, pedometer, RFID tags, RF sensors, compass, and more.
Computer vision techniques and the information taken from these sensors are used to
determine the user's location and orientation within his or her environment. On
the other hand, Guide Dog uses only an RGB-D camera to get information about its
environment and then uses well-known computer vision techniques to infer what is
contained in the environment. Furthermore, Guide Dog cannot map out its
environment and its hardware setup is not comfortably portable, which are
features that SWAN provides.

Guide Dog's obstacle detection feature is also similar to the project Kinecthesia
\cite{kinecthesia-website}. This group created a belt with a mounted RGB-D
camera to detect obstacles in front of a user. The belt uses vibration
motors to alert the user to the location of obstacles. As the user gets closer
to an obstacle, the intensity of the vibrations also get stronger. However, in
Guide Dog, obstacle detection does not communicate distance to the user through
the audio interface. Kinecthesia also divides the
scene into regions just like Guide Dog's audio interface. In the case of
Kinecthesia, they mapped each region to a vibration motor on the belt whereas
Guide Dog uses regions to better communicate the direction of objects through
the audio interface. See section \ref{sec:eval-audio} for more details on Guide
Dog's use of regions.

Also in the vein of indoor navigation using an RGB-D camera, there is a student
project called Navigational Aids for the Visually Impaired (NAVI)
\cite{navi-website} from the University of Konstanz. This project uses a
head-mounted camera, a specially designed backpack, and vibration motors in a
belt to guide a user through a building. Unlike Guide Dog, NAVI uses a voice
guidance system that gives speech commands. It guides a user through
hallways and alerts the user of obstacles.
This project is unique because the creators placed markers inside their
building for NAVI to navigate with.


\section{Technical Details}
\label{sec:technical}

Guide Dog is composed of the three separate components described below: destination detection
(see Section~\ref{sec:technical-dest}),
obstacle detection (see Section~\ref{sec:technical-obs}), and the audio
interface (see Section~\ref{sec:technical-audio}).

\begin{figure}
\includeimage{color.pdf}{8cm}{bb=0 0 474 474}
\caption{A sample scene, viewed in color. The destination object is the pink box
  in the middle. The two stacks of brown boxes on the left and right are
  obstacles.}
\label{fig:color}
\end{figure}

\subsection{Destination Detection}
\label{sec:technical-dest}

\begin{figure}
\includeimage{destination.pdf}{8cm}{bb=0 0 474 474}
\caption{A similarity view of Figure~\ref{fig:color} showing regions of the scene with a high similarity
to the color of the destination object. The green dot indicates the predicted location of the destination.}
\label{fig:destination}
\end{figure}

\subsubsection{Overview}
\label{sec:technical-dest-overview}

Guide Dog's destination detection component is responsible for identifying the
destination object in the scene, and determining its position in space
relative to the camera. Guide Dog was originally imagined as a system that
would be able to detect a wide variety of objects, but we were forced to
build a simpler prototype due to time constraints. As a result, the component 
is only able to detect objects of a solid color that contrast highly from the 
scene. Functionality is provided for re-calibrating the system for a new color, 
or for different lighting or exposure conditions. 

\subsubsection{Operation}
\label{sec:technical-dest-op}

To set up the destination detection component the user must first perform a
calibration step in which the color of the destination object is measured.
This measurement can be performed on the fly, by placing the
destination object in front of the camera, and then measuring the average 
color within the center region of the image. Once the color of the destination
object is known, we compute a similarity image where each pixel represents the 
similarity of an input pixel's color to the measured destination object color. Once 
this image is obtained, we binarize the pixel values by applying a 
user-adjustable threshold. The result is a binary image like
Figure~\ref{fig:destination}. A \emph{blob detection} algorithm is then run over the 
binary image, which finds the largest connected components by treating it as an 
8-connected graph. If multiple blobs are detected, the largest one is used.

Since the blobs describe a pixel location in the image, we must now convert 
this location into a real-world position relative to the camera. We take
advantage of PCL~\cite{pcl-website}, which provides camera-relative XYZ coordinates
for each pixel. We use this to compute the average XYZ position for pixels
within the largest blob, which then is returned as the final estimate of the location
of the destination object. 


\subsubsection{Implementation Details}
\label{sec:technical-dest-impl}

The similarity metric we use to compute the binary image in Figure~\ref{fig:destination}
treats all pixels as vectors in 3-space. In order
to compute these similarity values, we first normalize all pixel vectors
to have unit length. We then represent the difference between two pixels using the
magnitude of their difference vector. For example, if \emph{c} is the color of a
given pixel, and \emph{d} is the color of the destination object, then this gives the 
difference \begin{math} \textit{diff} = c - d \end{math} and the similarity 
\begin{math}\textit{sim} = 1 - ||\textit{diff}|| \end{math}.

In order to make the system more robust to noise, we apply a Gaussian blur
filter to the similarity image. We found a filter kernel size of 10 to 15 to work 
best for our application. Since the similarity computation tends
to amplify the appearance of noise, and also because noise tends to
negatively effect blob detection, this added step helps significantly increase
the robustness of the component.

After computing a similarity image in this fashion, we binarize the image
using a simple user-set threshold, and pass the binary image to the OpenCV~\cite{opencv-website}
\emph{SimpleBlobDetector}. This routine reliably detects connected components
within the image, but requires many parameters to be set properly, such as
minimum and maximum  blob size, convexity, and separation. After detecting blobs, the 
largest blob is used to compute a centroid in 3-space using pixels belonging 
to the blob.

\subsection{Obstacle Detection}
\label{sec:technical-obs}

\begin{figure}
\includeimage{obstacle.pdf}{5cm}{bb=0 0 300 300}
\caption{A view of the scene in Figure~\ref{fig:color}. This is a top down view.
  The user's location is shown with the blue X. The obstacles are shown labeled
  with red dots and the destination is labeled with a green dot.}
\label{fig:obstacle}
\end{figure}

\subsubsection{Overview}
\label{sec:technical-obs-overview}

The second component of Guide Dog is the component to detect obstacles. This
component must detect the obstacles near the user and convey the obstacles'
locations to the audio interface. Only obstacles near the user are detected in
order to avoid overwhelming the user. It is only important for the user to
know if he or she is about to run into an obstacle, not if there is obstacle on the
other side of the room.

\subsubsection{Operation}
\label{sec:technical-obs-op}

The obstacle detection algorithm looks only at the depth information from the
RGB-D camera. From this, it can see objects such as the floor, obstacles and
walls. Intuitively, all of the obstacles will be above the floor, while the floor
itself is not an obstacle. This leads to a simple algorithm to detect the
obstacles: just remove the floor and everything left is an obstacle.

\subsubsection{Implementation Details}
\label{sec:technical-obs-impl}

The obstacle detection algorithm views the points in a format created by the
Point Cloud Library (PCL)~\cite{pcl-website}. Each individual pixel viewed by
the camera is represented by an X coordinate, Y coordinate, depth and RGB color.
A data structure storing all of these points is called a ``point cloud'' and
represents the whole 3D environment that the camera can see.

The obstacle detection algorithm first looks at the point cloud and extracts the
plane of the floor. PCL has a built-in planar segmentation library function that
detects the largest plane in view (there is an issue if the floor is not the
largest plane in view, see Section~\ref{sec:eval-obs}).
The plane detection algorithm gives the \emph{a}, \emph{b}, \emph{c}, and
\emph{d} coefficients in the following planar equation:

\begin{math}
ax + by + cz + d = 0
\end{math}

This results in an equation representing the plane of the floor. The obstacle
detection algorithm then removes all points that lie within a certain threshold
of the floor plane. The threshold helps remove noise added by the camera and is
currently set to 10 centimeters. After removing the floor, only the obstacles
are left. However, the obstacles are represented in 3D space, which is not
necessary for obstacle detection. This is because the height of an obstacle does
not matter. If there is an obstacle at any height, this needs to be
communicated to the user. This allows the obstacle detection algorithm to
convert the 3D obstacles to 2D space. To do this, the 3D obstacle coordinates
are projected onto the floor plane. Then, the floor plane is rotated so that it
is level. This provides a simple way to analyze the obstacles and calculate
distances.

Once the obstacles have been extracted and projected into a 2D space, more
analysis is performed to detect their locations. The obstacles are processed
into a black and white image: a white pixel means there is part of an obstacle
at that location, a black pixel means that location is empty. An example of this
image is shown in Figure~\ref{fig:obstacle}. This image is then
passed off to the same blob detection algorithm described in Section~\ref{sec:technical-dest-impl}.
As is shown in the example image in
Figure~\ref{fig:obstacle}, each obstacle is represented as a contiguous block
of white pixels. The blob detection algorithm detects the contiguous blocks and
produces a single point for each blob, marked with a red dot in the example
figure.

At this point, the obstacle detection algorithm has the locations of all
obstacles. There are two more steps that must happen before the obstacles are
sent to the audio interface. First, the obstacle detection component must
communicate with the destination detection component. This is because
the obstacle detection component has no idea where the destination object is,
meaning the destination component will get detected and marked as an obstacle!
In order to prevent this, the obstacle detection component gets the coordinates
of the destination object from the destination detection component. It then
compares the coordinates of each obstacle it found with the coordinates of the
destination object. If any of the obstacle coordinates are close enough to the
destination, the obstacle detection component assumes that the obstacle must
actually be the destination and removes it from its list of obstacles. Second,
the obstacle detection component is only supposed to communicate obstacles that
are near to the user, not all obstacles. To do this, the obstacle detection
component simply removes any obstacles that are too far away from the user.

At this point, the remaining obstacles are ready to be passed off to the audio
interface so they can be communicated to the user.

\subsection{Audio Interface}
\label{sec:technical-audio}

\subsubsection{Overview}
\label{sec:technical-audio-overview}

%(Provide the context of what your system does. A drawing might be useful here.)
Because the intended user of Guide Dog may not have normal vision, the audio 
component is meant to provide guidance to the user without the need of sight. It
uses 3D audio cues to direct the user toward the destination and warn of any 
obstacles in the user's way. This means that the audio cues appear as though
they emanate from a particular point in 3D space that corresponds to the
destination or obstacle.

\subsubsection{Operation}
\label{sec:technical-audio-op}

%(Detailed description of the different functionalities and how they work. For
%instance, describe your tracking / smoothing / recognition algorithm.)
The audio system signals the location of the destination through the use of an
audio beacon that sounds like a series of synthesized echoing beeps. It is fairly
easy to distinguish changes in tone using the beeps and it is not a sound that
quickly gets annoying to the user. The user is expected to follow the direction
of the beacon in order to get to the destination. To communicate distance to the
destination, the audio system increases or decreases the pitch of the beacon as 
the user gets closer or further away, respectively. Once the user arrives within
a few feet from the destination, the system plays a jingle to tell the user that 
he or she has arrived. 

As for obstacles, the audio system uses a ding that sounds similar
to the one that is played when a car door is left ajar. Like with the beacon, 
the goal was to find a sound that was pleasant over moderate periods of time. 
Furthermore, the ding is clearly distinguishable from the beacon. Unlike the
destination beacon, the user is expected to
avoid the direction of the alert. Only obstacles that are within about five feet of the user get alerted.
This strategy allows the user enough room to correct his or her course while 
still being helpful. If the system alerted obstacles at any distance, there may be 
too many, which would overwhelm and confuse the user, or the ones that are 
particularly far would cause the user to correct his route at times when the 
path in front of him is actually clear.

The audio system defines five discrete regions that an object may be located 
within. Compared to playing the sound purely as though it were emanating from 
its actual location in 3D space, these regions help the system exaggerate the 
left and right directions when sounds are played back to the user.  This 
strategy gives a clearer indication of which direction an object is located in.
These regions are based on the angle of the object's location relative to the 
user. The five regions are defined as full left, partial left, in front, partial 
right and full right. Section~\ref{sec:eval-audio} describes the layout and 
development of the region system in greater detail.

\subsubsection{Implementation Details}
\label{sec:technical-audio-impl}

%(Some details on the software implementation.)
The audio system is abstracted away from the other parts of Guide Dog and
written as a class with member functions that the other systems can invoke. At 
the core of this class, known as SonicDog, is OpenAL~\cite{openal-website}, which
is an open source audio library typically used in video games to handle 3D
sound. Developing a separate class enabled quick development of the audio system
without affecting other parts of GuideDog because the clients only needed to be
concerned with inputting the correct and consistent coordinates of objects in 
the scene.

SonicDog provides a thread pool and job queue for the threads. The destination 
and obstacle systems fill the queue with sound sources and the threads pop them 
off and play a sound for each source. OpenAL uses the concepts of a listener, to 
define the coordinates of the person listening, and sources, to define the 
objects playing a sound. The listener and sources both have locations within the 
OpenAL world. SonicDog defines the listener at \emph{(0, 0, 0)} under the 
\emph{(x, y, z)} coordinate system. All other sound sources are placed relative 
to the listener by defining the \emph{x} and \emph{z} coordinates of the source 
since the other systems in Guide Dog reduce the world to a 2D plane. Positive 
\emph{x} denotes an object being to the right of the listener and negative 
\emph{x} as being to the left. Similarly, positive \emph{z} is in front and 
negative \emph{z} is behind. Furthermore, each source has a sound buffer, which 
defines the sound to be played for the source.

To signal destinations, the destination system registers the object with
SonicDog, which initializes the source and returns an identification number. 
The destination system then refers to this number when it wants to update the
new location of the registered object. Each time Guide Dog updates the location,
there is a subroutine in SonicDog that calculates the angle of the object 
relative to the user and then manipulates the internal OpenAL coordinates to 
place it in the correct region. Meanwhile, the thread responsible for playing 
the beacon continuously plays the beacon every 1.5 seconds and then 
recalculates the pitch of the beacon based on the current values it has for the 
position of the destination object. SonicDog handles obstacles differently.

Because obstacles are not tracked from frame to frame, SonicDog provides an
interface to alert a group of obstacles once. Each time the obstacle detection
system runs, it can pass in a list of obstacle locations to SonicDog. The same
subroutine runs to place the obstacles into their respective regions, a new
source is created in OpenAL, and then the source is pushed onto the job queue.
From there, a worker thread pulls them off one by one and tells OpenAL to play
the source. Afterwards, the thread cleans up all the memory allocated for the
source and goes back to pull more sources off the job queue or blocks if the 
queue is empty.

The math used to place an object in a specific region is straightforward. Given
the \emph{x} and \emph{z} coordinates of an object, SonicDog uses arctangent to
calculate the angle \begin{math}\theta\end{math} between the object and the
listener, and the Pythagorean theorem to calculate the distance \emph{d} from
the object to the listener. In the five-region design in Figure~\ref{fig:regions}, the
middle region is the arc considered in front of the user and is between 85 and
95 degrees. When \begin{math}85\leq\theta\leq95\end{math}, SonicDog sets the
internal coordinates to \emph{(0, 0, d)} so that OpenAL plays a sound that
appears as though it is coming from in front of the user. If
\begin{math}\theta>110\end{math}, SonicDog places the source at \emph{(-d, 0,
0)} because that region is considered the complete left. A similar operation
occurs for when \begin{math}\theta<70\end{math} and the object is on the
complete right. For the 15 degree arcs representing the partial left or right
regions, SonicDog uses the equation \begin{math}\phi=2\theta-140\end{math} to
translate the source to an angle \begin{math}\phi\end{math} in the OpenAL world
that is further left or right than the real world location. This makes the
changes in stereo more audible to the user. The effect is that the user
hears an emphasis of the sound in one ear instead of it being completely pushed
to that ear. This tells the user that the object is only located slightly to the
left or right instead it being ambiguous like in the three region design. The
corresponding internal coordinates would be set to \begin{math}(d\cos\phi, 0,
d\sin\phi)\end{math}. Section \ref{sec:eval-audio} goes into further details
about the regions used by SonicDog.

Since this is a multi-threaded system, the implementation required multiple
locks to synchronize access to the maps inside SonicDog that keep track of what
objects have been registered to it. In addition, condition variables are used to
synchronize the job queue. These allow for a thread to be blocked and not use 
system resources when the queue is empty and for broadcasts to wake up threads 
when the queue becomes non-empty.


\section{Evaluation and Results (\textasciitilde 2 pages)}
\label{sec:eval}

This section details the authors' experiences building the Guide Dog system:
what worked, what didn't work and why.

\subsection{Destination Detection}
\label{sec:eval-dest}

Although the authors originally intended to be able to detect a variety
of complex objects as destination targets, it quickly became apparent that
the difficulty of detecting an arbitrary object had been underestimated.
As a placeholder for a more sophisticated object detector, we implemented
a detector that could find and track an object so long as it stood out
clearly from its environment. This lead to the limitation that
destination objects work best when they are of a solid color. We found
that bright magenta tended to stand out best from typical interior
environments. 

Even with the above limitation, the authors found that the orientation
of the destination object, diffuse reflection, and differences in lighting
(even between two places in the same room) were factors that could lead
to a failure to detect the destination object. To mitigate these issues,
the authors experimented with various similarity metrics such as cosine
similarity, added blob detection for the similarity image, and used a
destination object with a round surface (e.g. a coffee can) to reduce
dependence on the orientation of the object. With these improvements,
the system was able to reliably detect the destination under normal
indoor lighting conditions. 



\subsection{Obstacle Detection}
\label{sec:eval-obs}

The first difficulty the authors encountered with the obstacle detection
component was rotating the 3D view of the environment such that the floor was
level. This was an important step in the process of converting the 3D
environment to a 2D top down view (described in
Section~\ref{sec:technical-obs-impl}). The authors first attempted a series of
matrix rotations and translations, but had little luck getting this to work.
Instead, the authors discovered a projection library function in PCL. This takes
a plane and projects all the points in the environment onto this plane. The
environment if first projected onto the ground plane, which flattens all of the
obstacles into 2D, on the ground plane. Then, this result is projected on to the
X-Z axis. This way the Y component is removed, resulting in a 2D image.

Another aspect of the obstacle detection component the authors experimented with
was actually identifying the locations of the obstacles. The first approach
simply split the environment into a left and right half. The obstacle detection
component would then identify the closest point on the left side and the
closest point on the right side as obstacles. This was a simplistic approach,
but didn't work well for a few reasons. First, it did not pay attention to the
size of the obstacles. For example, the image contains small amounts of noise
that appear as small obstacles. However, the obstacle detection component should
not include these as obstacles. Second, it did not group contiguous points into
a single obstacle. This is not a problem if the obstacles are both restricted to
only one side of the environment. However, it was often the case that an
obstacle would be right in the center. So when the obstacle detection component
searched for the two closest points on the left and right it would settle upon
two points that were actually on the same obstacle.

To address these problems, the authors switched to using the blob detection
approach described in Section~\ref{sec:technical-obs-impl}. The blob detection
algorithm has parameters to set for the minimum and maximum allowable size of
the obstacles. By setting the minimum high enough, this filters out the
small areas of noise that weren't actually obstacles. Further, since the blob
detection clumps together contiguous areas there is no concern of erroneously
identifying a single physical obstacle as multiple obstacles.

One current issue the obstacle detection component has it when it fails to
detect the ground plane. The planar segmentation algorithm the obstacle
detection component uses doesn't actually detect the ground, it detects the
largest plane in view. Often times, this will be the ground. However, if the
camera is pointed too high or obstacles are blocking most of the ground it is
likely that the detected plane will not actually be the ground. The obstacle
detection component currently assumes that the largest plane in view will be the
ground plane. This results in an incorrect 2D top down view of the environment
when the largest plane detected is not actually the ground and therefore
incorrect detection of the obstacles. One way this could be handled is by
restricting the legal angles of the ground plane. If the detected ground plane
is outside of the boundaries, the obstacle detection component could remove the
points in the detected plane and then run the plane detection algorithm again
until it finds the plane within the legal boundaries of a ground plane.

The obstacle detection component also fails if the plane detection algorithm
can't find a plane at all. In this case, the obstacle detection component
currently stops and does not report any obstacles. One way to fix this would be
to use the ground plane equation from the previous camera frame. This would not
lead to the best results however, since the camera angle will naturally change
slightly between frames as the user is moving through the environment. Using an
old ground plane would likely remove some of the floor, but leave parts of the
floor that don't match up exactly with the old ground plane. This would result
in parts of the floor appearing like obstacles, which would be confusing to the
user.


\subsection{Audio Interface}
\label{sec:eval-audio}


\section{Discussion}
\label{sec:discussion}

Guide Dog works well within the limits of what its prototype components were
designed for. However, as a complete system, Guide Dog falls short of the 
vision its authors had originally intended for it. It is only able to track
brightly colored objects, it requires that the ground plane be within view
at all times, and it cannot handle the case where the destination object is 
outside the viewing range of the camera. 

The most obvious next step for the destination detection component is to
replace it with a more sophisticated, machine-learning based algorithm that
is able to learn and detect common household objects with high precision.
Such an algorithm would probably be based on feature learning, such as
previous object-detection work~\cite{lai_icra12}.
This would enable the system to detect and guide the user to real-world
objects of practical interest to the user, greatly increasing the
utility of the system.

An improvement of even greater scope, which would benefit the system as
a whole, would be the addition of a 3D-mapping component and a system
for localizing the camera's position within a pre-existing 3D map~\cite{Du:2011:IMI:2030112.2030123}. This
would allow the system to be placed into a previously-mapped space,
determine its location, and guide the user to a distant point well
outside the line of sight of the camera. A reasonably sophisticated 
localization component would remove the need to see the floor at all
times, enabling obstacle detection based on the 3D map, rather than 
planar projection. It would also allow shortest paths to be computed
through a 3D map, enabling audio guidance with more detailed cues for 
guiding the user along the path (e.g. synthetic speech).

The open source support for RGB-D image processing was impressive. Many of the
``hard'' parts of the project were already implemented. For example, PCL's
planar segmentation worked perfectly to detect the ground plane for obstacle
detection. OpenCV's blob detection was simple to set up and use and was taken
advantage of in both destination and obstacle detection.

In the future, it would be beneficial to have a fully integrated system
completed earlier. With a fully integrated system it would be easier to test the
audio interface in a real world environment, which would allow for better
iteration and improvement of the audio system.


\section{Conclusion}
\label{sec:conclusion}

In the domain of RGB-D camera software, the space of applications in which the
camera is mounted to the user remains relatively unexplored. Guide Dog attempts
to explore this space as a prototype guidance tool for users who are either
visually impaired or unable to see for other reasons. The Guide Dog system
described here makes many simplifying assumptions about its environment and
use cases, but research in related domains is very encouraging. Advances in
object detection and labeling could be incorporated to make Guide Dog's
destination guidance far more robust, and recent results in 3D mapping and
reconstruction could be incorporated to remove Guide Dog from the confines
of its local field of view. If incorporated into the system, these items
would make Guide Dog a useful tool for practical situations, and would bring
Guide Dog much closer to the initial vision of the authors. 

Guide Dog's implementation is publicly available to use and build
upon~\cite{guidedog-website}.


%
% The following two commands are all you need in the
% initial runs of your .tex file to
% produce the bibliography for the citations in your paper.
\bibliographystyle{abbrv}
\bibliography{guide-dog}  % guide-dog.bib is the name of the Bibliography in this case
% You must have a proper ".bib" file
%  and remember to run:
% latex bibtex latex latex
% to resolve all references
%
% ACM needs 'a single self-contained file'!
%
\balancecolumns
% That's all folks!
\end{document}
