% THIS IS SIGPROC-SP.TEX - VERSION 3.1
% WORKS WITH V3.2SP OF ACM_PROC_ARTICLE-SP.CLS
% APRIL 2009
%
% It is an example file showing how to use the 'acm_proc_article-sp.cls' V3.2SP
% LaTeX2e document class file for Conference Proceedings submissions.
% ----------------------------------------------------------------------------------------------------------------
% This .tex file (and associated .cls V3.2SP) *DOES NOT* produce:
%       1) The Permission Statement
%       2) The Conference (location) Info information
%       3) The Copyright Line with ACM data
%       4) Page numbering
% ---------------------------------------------------------------------------------------------------------------
% It is an example which *does* use the .bib file (from which the .bbl file
% is produced).
% REMEMBER HOWEVER: After having produced the .bbl file,
% and prior to final submission,
% you need to 'insert'  your .bbl file into your source .tex file so as to provide
% ONE 'self-contained' source file.
%
% Questions regarding SIGS should be sent to
% Adrienne Griscti ---> griscti@acm.org
%
% Questions/suggestions regarding the guidelines, .tex and .cls files, etc. to
% Gerald Murray ---> murray@hq.acm.org
%
% For tracking purposes - this is V3.1SP - APRIL 2009

\documentclass{acm_proc_article-sp}
\usepackage{hyperref}

\begin{document}

\conferenceinfo{CSE 481 O Final Report,}{Autumn 2012}

\newcommand{\includeimage}[3]{
\begin{center}
%\ifhevea\imgsrc{#1.png}\else
\resizebox{!}{#2}{\includegraphics[#3]{images/#1}}
\vspace{-1.5\baselineskip}
%\fi
\end{center}}

\title{Guide Dog: An Audio Guidance Tool}

%
% You need the command \numberofauthors to handle the 'placement
% and alignment' of the authors beneath the title.
%
% For esthetic reasons, we recommend 'three authors at a time'
% i.e. three 'name/affiliation blocks' be placed beneath the title.
%
% NOTE: You are NOT restricted in how many 'rows' of
% "name/affiliations" may appear. We just ask that you restrict
% the number of 'columns' to three.
%
% Because of the available 'opening page real-estate'
% we ask you to refrain from putting more than six authors
% (two rows with three columns) beneath the article title.
% More than six makes the first-page appear very cluttered indeed.
%
% Use the \alignauthor commands to handle the names
% and affiliations for an 'esthetic maximum' of six authors.
% Add names, affiliations, addresses for
% the seventh etc. author(s) as the argument for the
% \additionalauthors command.
% These 'additional authors' will be output/set for you
% without further effort on your part as the last section in
% the body of your article BEFORE References or any Appendices.

\numberofauthors{3} %  in this sample file, there are a *total*
% of EIGHT authors. SIX appear on the 'first-page' (for formatting
% reasons) and the remaining two appear in the \additionalauthors section.
%
%\author{
% You can go ahead and credit any number of authors here,
% e.g. one 'row of three' or two rows (consisting of one row of three
% and a second row of one, two or three).
%
% The command \alignauthor (no curly braces needed) should
% precede each author name, affiliation/snail-mail address and
% e-mail address. Additionally, tag each line of
% affiliation/address with \affaddr, and tag the
% e-mail address with \email.
%

\author{
Robert Bart
\qquad
Minh-Quan Nguyen
\qquad
Eric Spishak\\
%
{\normalsize
University of Washington}\\
\url{{rbart,nguyenmq,espishak}@cs.washington.edu}
}

\maketitle
\begin{abstract}
RGB-D cameras, like the Microsoft Kinect, are gaining popularity. Currently, the
technology is used mostly in games. However, the technology is relatively young
and the applications for it are still unknown.

Guide Dog is an experimental application of an RGB-D camera, that
uses audio to direct a user to a destination, while avoiding obstacles. The
system relies on an RGB-D camera mounted on the user to detect the features of
the environment, which are then communicated by audio to the user. It uses
various audio tones to convey the directions and distances of the destination
and obstacles. Guide Dog could be used by a blind person to navigate through a
room or to help a user find his or her keys in the dark.

This paper describes the authors' experiences designing, building and improving
this system. Included is a detailed description of the technologies used and the
paths the authors took to get a final, working product.

Guide Dog's implementation is available online for anyone to try and improve
upon.
\end{abstract}

\keywords{RGB-D cameras, accessibility tools} % NOT required for Proceedings

\section{Introduction (\textasciitilde 1st page)}
\label{sec:intro}

Introduce the basic idea of your project, the key motivation (why should people
care), high level background information, and what you plan to describe in
this paper. Feel free to include a picture / drawing.

Citation\cite{salas:calculus}


\section{Related Work (\textasciitilde 1 column)}
\label{sec:related}

Describe other research projects, commercial products, patents, etc that are
related to your project and how they differ from your work. Also, if your
work is based on previous techniques, describe them here. Should be about 1
column.


% Force column break to prevent section header from being in previous column
\vfill\eject

\section{Technical Details}
\label{sec:technical}

Guide Dog is composed of the three separate components described below: destination detection
(see Section~\ref{sec:technical-dest}),
obstacle detection (see Section~\ref{sec:technical-obs}), and the audio
interface (see Section~\ref{sec:technical-audio}).

\begin{figure}
\includeimage{color.pdf}{8cm}{bb=0 0 474 474}
\caption{A sample scene, viewed in color. The destination object is the pink box
  in the middle. The two stacks of brown boxes on the left and right are
  obstacles.}
\label{fig:color}
\end{figure}

\subsection{Destination Detection}
\label{sec:technical-dest}

\begin{figure}
\includeimage{destination.pdf}{8cm}{bb=0 0 474 474}
\caption{A similarity view of Figure~\ref{fig:color} showing regions of the scene with a high similarity
to the color of the destination object. The green dot indicates the predicted location of the destination.}
\label{fig:destination}
\end{figure}

\subsubsection{Overview}
\label{sec:technical-dest-overview}

Guide Dog's destination detection component is responsible for identifying the
destination object in the scene, and determining its position in space
relative to the camera. Guide Dog was originally imagined as a system that
would be able to detect a wide variety of objects, but we were forced to
build a simpler prototype due to time constraints. As a result, the component 
is only able to detect objects of a solid color that contrast highly from the 
scene. Functionality is provided for re-calibrating the system for a new color, 
or for different lighting or exposure conditions. 

\subsubsection{Operation}
\label{sec:technical-dest-op}

To set up the destination detection component the user must first perform a
calibration step in which the color of the destination object is measured.
This measurement can be performed on the fly, by placing the
destination object in front of the camera, and then measuring the average 
color within the center region of the image. Once the color of the destination
object is known, we compute a similarity image where each pixel represents the 
similarity of an input pixel's color to the measured destination object color. Once 
this image is obtained, we binarize the pixel values by applying a 
user-adjustable threshold. The result is a binary image like
Figure~\ref{fig:destination}. A \emph{blob detection} algorithm is then run over the 
binary image, which finds the largest connected components by treating it as an 
8-connected graph. If multiple blobs are detected, the largest one is used.

Since the blobs describe a pixel location in the image, we must now convert 
this location into a real-world position relative to the camera. We take
advantage of PCL~\cite{pcl-website}, which provides camera-relative XYZ coordinates
for each pixel. We use this to compute the average XYZ position for pixels
within the largest blob, which then is returned as the final estimate of the location
of the destination object. 


\subsubsection{Implementation Details}
\label{sec:technical-dest-impl}

The similarity metric we use to compute the binary image in Figure~\ref{fig:destination}
treats all pixels as vectors in 3-space. In order
to compute these similarity values, we first normalize all pixel vectors
to have unit length. We then represent the difference between two pixels using the
magnitude of their difference vector. For example, if \emph{c} is the color of a
given pixel, and \emph{d} is the color of the destination object, then this gives the 
difference \begin{math} \textit{diff} = c - d \end{math} and the similarity 
\begin{math}\textit{sim} = 1 - ||\textit{diff}|| \end{math}.

In order to make the system more robust to noise, we apply a Gaussian blur
filter to the similarity image. We found a filter kernel size of 10 to 15 to work 
best for our application. Since the similarity computation tends
to amplify the appearance of noise, and also because noise tends to
negatively effect blob detection, this added step helps significantly increase
the robustness of the component.

After computing a similarity image in this fashion, we binarize the image
using a simple user-set threshold, and pass the binary image to the OpenCV~\cite{opencv-website}
\emph{SimpleBlobDetector}. This routine reliably detects connected components
within the image, but requires many parameters to be set properly, such as
minimum and maximum  blob size, convexity, and separation. After detecting blobs, the 
largest blob is used to compute a centroid in 3-space using pixels belonging 
to the blob.

\subsection{Obstacle Detection}
\label{sec:technical-obs}

\begin{figure}
\includeimage{obstacle.pdf}{5cm}{bb=0 0 300 300}
\caption{A view of the scene in Figure~\ref{fig:color}. This is a top down view.
  The user's location is shown with the blue X. The obstacles are shown labeled
  with red dots and the destination is labeled with a green dot.}
\label{fig:obstacle}
\end{figure}

\subsubsection{Overview}
\label{sec:technical-obs-overview}

The second component of Guide Dog is the component to detect obstacles. This
component must detect the obstacles near the user and convey the obstacles'
locations to the audio interface. Only obstacles near the user are detected in
order to avoid overwhelming the user. It is only important for the user to
know if he or she is about to run into an obstacle, not if there is obstacle on the
other side of the room.

\subsubsection{Operation}
\label{sec:technical-obs-op}

The obstacle detection algorithm looks only at the depth information from the
RGB-D camera. From this, it can see objects such as the floor, obstacles and
walls. Intuitively, all of the obstacles will be above the floor, while the floor
itself is not an obstacle. This leads to a simple algorithm to detect the
obstacles: just remove the floor and everything left is an obstacle.

\subsubsection{Implementation Details}
\label{sec:technical-obs-impl}

The obstacle detection algorithm views the points in a format created by the
Point Cloud Library (PCL)~\cite{pcl-website}. Each individual pixel viewed by
the camera is represented by an X coordinate, Y coordinate, depth and RGB color.
A data structure storing all of these points is called a ``point cloud'' and
represents the whole 3D environment that the camera can see.

The obstacle detection algorithm first looks at the point cloud and extracts the
plane of the floor. PCL has a built-in planar segmentation library function that
detects the largest plane in view (there is an issue if the floor is not the
largest plane in view, see Section~\ref{sec:eval-obs}).
The plane detection algorithm gives the \emph{a}, \emph{b}, \emph{c}, and
\emph{d} coefficients in the following planar equation:

\begin{math}
ax + by + cz + d = 0
\end{math}

This results in an equation representing the plane of the floor. The obstacle
detection algorithm then removes all points that lie within a certain threshold
of the floor plane. The threshold helps remove noise added by the camera and is
currently set to 10 centimeters. After removing the floor, only the obstacles
are left. However, the obstacles are represented in 3D space, which is not
necessary for obstacle detection. This is because the height of an obstacle does
not matter. If there is an obstacle at any height, this needs to be
communicated to the user. This allows the obstacle detection algorithm to
convert the 3D obstacles to 2D space. To do this, the 3D obstacle coordinates
are projected onto the floor plane. Then, the floor plane is rotated so that it
is level. This provides a simple way to analyze the obstacles and calculate
distances.

Once the obstacles have been extracted and projected into a 2D space, more
analysis is performed to detect their locations. The obstacles are processed
into a black and white image: a white pixel means there is part of an obstacle
at that location, a black pixel means that location is empty. An example of this
image is shown in Figure~\ref{fig:obstacle}. This image is then
passed off to the same blob detection algorithm described in Section~\ref{sec:technical-dest-impl}.
As is shown in the example image in
Figure~\ref{fig:obstacle}, each obstacle is represented as a contiguous block
of white pixels. The blob detection algorithm detects the contiguous blocks and
produces a single point for each blob, marked with a red dot in the example
figure.

At this point, the obstacle detection algorithm has the locations of all
obstacles. There are two more steps that must happen before the obstacles are
sent to the audio interface. First, the obstacle detection component must
communicate with the destination detection component. This is because
the obstacle detection component has no idea where the destination object is,
meaning the destination component will get detected and marked as an obstacle!
In order to prevent this, the obstacle detection component gets the coordinates
of the destination object from the destination detection component. It then
compares the coordinates of each obstacle it found with the coordinates of the
destination object. If any of the obstacle coordinates are close enough to the
destination, the obstacle detection component assumes that the obstacle must
actually be the destination and removes it from its list of obstacles. Second,
the obstacle detection component is only supposed to communicate obstacles that
are near to the user, not all obstacles. To do this, the obstacle detection
component simply removes any obstacles that are too far away from the user.

At this point, the remaining obstacles are ready to be passed off to the audio
interface so they can be communicated to the user.

\subsection{Audio Interface}
\label{sec:technical-audio}

\subsubsection{Overview}
\label{sec:technical-audio-overview}

%(Provide the context of what your system does. A drawing might be useful here.)
Because the intended user of Guide Dog may not have normal vision, the audio 
component is meant to provide guidance to the user without the need of sight. It
uses 3D audio cues to direct the user toward the destination and warn of any 
obstacles in the user's way. This means that the audio cues appear as though
they emanate from a particular point in 3D space that corresponds to the
destination or obstacle.

\subsubsection{Operation}
\label{sec:technical-audio-op}

%(Detailed description of the different functionalities and how they work. For
%instance, describe your tracking / smoothing / recognition algorithm.)
The audio system signals the location of the destination through the use of an
audio beacon that sounds like a series of synthesized echoing beeps. It is
fairly easy to distinguish changes in tone using the beeps and it is not a sound
that quickly gets annoying to the user. The user is expected to follow the
direction of the beacon in order to get to the destination. To communicate
distance to the destination, the audio system increases or decreases the pitch
of the beacon as the user gets closer or further away, respectively. Once the
user arrives within a few feet from the destination, the system plays a jingle
to tell the user that he or she has arrived. 

As for obstacles, the audio system uses a ding that sounds similar to the one
that is played when a car door is left ajar. Like with the beacon, the goal was
to find a sound that was pleasant over moderate periods of time.  Furthermore,
the ding is clearly distinguishable from the beacon. Unlike the destination
beacon, the user is expected to avoid the direction of the alert.

The audio system defines five discrete regions that an object may be located
within. Compared to playing the sound purely as though it were emanating from
its actual location in 3D space, these regions help the system exaggerate the
left and right directions when sounds are played back to the user.  This
strategy gives a clearer indication of which direction an object is located in.
These regions are based on the angle of the object's location relative to the
user. The five regions are defined as full left, partial left, in front, partial
right and full right. Section~\ref{sec:eval-audio} describes the layout and
development of the region system in greater detail.

\subsubsection{Implementation Details}
\label{sec:technical-audio-impl}

%(Some details on the software implementation.)
The audio system is abstracted away from the other parts of Guide Dog and
written as a class with member functions that the other systems can invoke. At 
the core of this class, known as SonicDog, is OpenAL~\cite{openal-website}, which
is an open source audio library typically used in video games to handle 3D
sound. Developing a separate class enabled quick development of the audio system
without affecting other parts of GuideDog because the clients only needed to be
concerned with inputting the correct and consistent coordinates of objects in 
the scene.

SonicDog has a thread pool and job queue for the threads. The destination
and obstacle systems fill the queue with sound sources and the threads pop them 
off and play a sound for each source. OpenAL uses the concepts of a listener, to 
define the coordinates of the person listening, and sources, to define the 
objects playing a sound. The listener and sources both have locations within the 
OpenAL world. SonicDog defines the listener at \emph{(0, 0, 0)} under the 
\emph{(x, y, z)} coordinate system. All other sound sources are placed relative 
to the listener by defining the \emph{x} and \emph{z} coordinates of the source 
since the other systems in Guide Dog reduce the world to a 2D plane. Positive 
\emph{x} denotes an object being to the right of the listener and negative 
\emph{x} as being to the left. Similarly, positive \emph{z} is in front and 
negative \emph{z} is behind. Furthermore, each source has a sound buffer, which 
defines the sound to be played for the source.

To signal destinations, the destination system registers the object with
SonicDog, which initializes the source and returns an identification number. 
The destination system then refers to this number when it wants to update the
new location of the registered object. Each time Guide Dog updates the location,
there is a subroutine in SonicDog that calculates the angle of the object 
relative to the user and then manipulates the internal OpenAL coordinates to 
place it in the correct region. Meanwhile, the thread responsible for playing 
the beacon continuously plays the beacon every 1.5 seconds and then 
recalculates the pitch of the beacon based on the current values it has for the 
position of the destination object. SonicDog handles obstacles differently.

Because obstacles are not tracked from frame to frame, SonicDog provides an
interface to alert a group of obstacles once. Each time the obstacle detection
system runs, it can pass in a list of obstacle locations to SonicDog. The same
subroutine runs to place the obstacles into their respective regions, a new
source is created in OpenAL, and then the source is pushed onto the job queue.
From there, a worker thread pulls them off one by one and tells OpenAL to play
the source. Afterwards, the thread cleans up all the memory allocated for the
source and goes back to pull more sources off the job queue or blocks if the 
queue is empty.

The math used to place an object in a specific region is straightforward. Given
the \emph{x} and \emph{z} coordinates of an object, SonicDog uses arctangent to
calculate the angle \begin{math}\theta\end{math} between the object and the
listener, and the Pythagorean theorem to calculate the distance \emph{d} from
the object to the listener. In the five-region design in Figure~\ref{fig:regions}, the
middle region is the arc considered in front of the user and is between 85 and
95 degrees. When \begin{math}85\leq\theta\leq95\end{math}, SonicDog sets the
internal coordinates to \emph{(0, 0, d)} so that OpenAL plays a sound that
appears as though it is coming from in front of the user. If
\begin{math}\theta>110\end{math}, SonicDog places the source at \emph{(-d, 0,
0)} because that region is considered the complete left. A similar operation
occurs for when \begin{math}\theta<70\end{math} and the object is on the
complete right. For the 15 degree arcs representing the partial left or right
regions, SonicDog uses the equation \begin{math}\phi=2\theta-140\end{math} to
translate the source to an angle \begin{math}\phi\end{math} in the OpenAL world
that is further left or right than the real world location. This makes the
changes in stereo more audible to the user. The effect is that the user
hears an emphasis of the sound in one ear instead of it being completely pushed
to that ear. This tells the user that the object is only located slightly to the
left or right instead it being ambiguous like in the three region design. The
corresponding internal coordinates would be set to \begin{math}(d\cos\phi, 0,
d\sin\phi)\end{math}. Section \ref{sec:eval-audio} goes into further details
about the regions used by SonicDog.

Since this is a multi-threaded system, the implementation required multiple
locks to synchronize access to the maps inside SonicDog that keep track of what
objects have been registered to it. In addition, condition variables are used to
synchronize the job queue. These allow for a thread to be blocked and not use 
system resources when the queue is empty and for broadcasts to wake up threads 
when the queue becomes non-empty.


\section{Evaluation and Results}
\label{sec:eval}

\subsection{Destination Detection}
\label{sec:eval-dest}

Although the authors originally intended to be able to detect a variety
of complex objects as destination targets, it quickly became apparent that
the difficulty of detecting an arbitrary object had been underestimated.
As a placeholder for a more sophisticated object detector, we implemented
something that could find and track an object as long as it stood out
clearly from its environment. This led to the limitation that
destination objects work best when they are of a bright, solid color. We found
that a magenta color tended to stand out best from typical interior
environments. 

Even with the above limitation, the authors found that the orientation
of the destination object, diffuse reflection, and differences in lighting
(even between two places in the same room) were factors that could lead
to a failure to detect the destination object. 

To migitage these issues, the authors experimented with different similarity
metrics, blob detection, and different colors and shapes for the destination
object. In addition to the similarity metric described in 
Section~\ref{sec:technical-dest-impl}, the authors also tried using
the cosine similarity of color vectors, but found the difference magnitude
approach to perform better in practice. The addition of blob detection
allowed the system to detect the destination object under normal variation
in room lighting. The authors also experimented with various colors and
shapes for the destination object, and found that an object with a rounded
face (such as a cylinder or ball) helped to reduce the effect of diffuse
reflections. This lead to the use of a coffee can wrapped with matte pink paper
as the final destination object. With these improvements, the system was able 
to reliably detect the destination under normal indoor lighting conditions.

\subsection{Obstacle Detection}
\label{sec:eval-obs}

The first difficulty the authors encountered with the obstacle detection
component was rotating the 3D view of the environment such that the floor was
level. This was an important step in the process of converting the 3D
environment to a 2D top down view (described in
Section~\ref{sec:technical-obs-impl}). The authors first attempted a series of
matrix rotations and translations, but had little luck getting this to work.
Instead, the authors discovered a projection library function in PCL. This takes
a plane and projects all the points in the environment onto this plane. The
environment is first projected onto the ground plane, which flattens all of the
obstacles into 2D on the ground plane. Then, this result is projected on to the
X-Z axis. This way the Y component is removed, resulting in a 2D image.

Another aspect of the obstacle detection component the authors experimented with
was actually identifying the locations of the obstacles. The first approach
simply split the environment into a left and right half. The obstacle detection
component would then identify the closest point on the left side and the
closest point on the right side as obstacles. This was a simplistic approach,
but didn't work well for a few reasons. First, it incorrectly identified noise in
the image as obstacles. For example, the images sometimes contain small amounts of noise
that appear as small obstacles. Second, it did not group contiguous points into
a single obstacle. This is not a problem if the obstacles are both restricted to
only one side of the environment. However, it was often the case that an
obstacle would be directly in the center. So when the obstacle detection component
searched for the two closest points on the left and right it would settle upon
two points that were actually part of the same obstacle.

To address these problems, the authors switched to using the blob detection
approach described in Section~\ref{sec:technical-obs-impl}. The blob detection
algorithm has parameters to set for the minimum and maximum allowable size of
the obstacles. By setting the minimum high enough, this filters out the
small areas of noise that weren't actually obstacles. Further, since the blob
detection clumps together contiguous areas, there is no concern of erroneously
identifying a single physical obstacle as multiple obstacles.

One current issue the obstacle detection component is when it fails to
detect the ground plane. The planar segmentation algorithm the obstacle
detection component uses doesn't actually detect the ground, it detects the
largest plane in view. Often times, this will be the ground. However, if the
camera is pointed too high or if obstacles are blocking most of the ground, it is
likely that the detected plane will not actually be the ground. The obstacle
detection component currently assumes that the largest plane in view will be the
ground plane. This results in an incorrect 2D top down view of the environment
when the largest plane detected is not actually the ground and therefore
incorrect detection of the obstacles. One way this could be handled is by
restricting the legal angles of the ground plane. If the detected ground plane
is outside of the boundaries, the obstacle detection component could remove the
points in the detected plane and then run the plane detection algorithm again
until it finds the plane within the legal boundaries of a ground plane.

The obstacle detection component also fails if the plane detection algorithm
can't find a plane at all. In this case, the obstacle detection component
currently stops and does not report any obstacles. One way to fix this would be
to use the ground plane equation from the previous camera frame. This would not
lead to the best results however, since the camera angle will naturally change
slightly between frames as the user is moving through the environment. Using an
old ground plane would likely remove some of the floor, but leave parts of the
floor that don't match up exactly with the old ground plane. This would result
in parts of the floor appearing like obstacles, which would be confusing to the
user.


\subsection{Audio Interface}
\label{sec:eval-audio}

\begin{figure}
\includeimage{vsim.pdf}{6cm}{bb=0 0 371 371}
\caption{A program to simulate the expected functionality of Guide Dog, which was
used to aid development of the audio system. The user is represented by the
while box, the destination the pink box, and obstacles are the green boxes. The
user could move around in the environment and interact with the audio interface.}
\label{fig:vsim}
\end{figure}

Guide Dog's audio interface was initially imagined to be one that used voice
commands like in a GPS navigation system. However, such a system would require
mapping and path finding within the scene, which would have been too complex
given the time constraints. Furthermore, people are very capable of finding
their own paths given a few hints about where they need to be going and what
may be in their way. Thus, Guide Dog adopted a system that gives audio hints to
the user instead.

To aid development of the audio system, the authors created a program to
simulate Guide Dog's functionality. It began as a terminal program with no
visual output that enabled the user to blindly traverse the landscape while
following the beacon and avoiding obstacles. Later on, a new simulator with 
visual output was created based off the tutorial by~\cite{openal-tutorial} using
OpenGL~\cite{opengl-website}. Figure~\ref{fig:vsim} gives a description of the 
\emph{vsim} program. With it, the authors could test what strategies worked and
verify that any calculations made in SonicDog were correct independent of the
other systems in Guide Dog.

\begin{figure}
\includeimage{regions.pdf}{6cm}{bb=0 0 600 450}
\caption{The division of the camera field of view into discrete regions. Each
region exaggerates its defined direction to give the user a better sense of the
location of an object. The red square represents the camera's position and the
red lines represent the extent of its 58 degree field of view. The black lines
delineate each region. On the left is the first implementation of regions and
on the right is the second.}
\label{fig:regions}
\end{figure}

The first version of the audio system played audio cues as though they were
actually emanating from an object's real 3D position using stereo sound. However,
users could only get an approximate indication of an object's location because
of the subtle changes in the audio direction. The simulation demonstrated that
it was possible to navigate using this method, but the authors wanted a system
with feedback that was more clear. Thus, to further clarify the direction of an 
object, the audio system continues to use stereo sound, but divides the field of
view of the camera into discrete regions that exaggerate the front, left, and 
right directions.

There were two designs implemented for the regions. The angles for each region 
can been seen in Figure~\ref{fig:regions}. The first used three regions 
representing left, front, and right. Since the Asus Xtion camera has a 58 degree 
horizontal field of vision~\cite{xtion-website}, the audio system considers any 
object that falls within the middle 20 degrees of the view to be in front. The 
user will hear the audio cues with equal weight in both his or her ears. Any 
objects that are located outside those middle 20 degrees are considered to be to 
the full left or right. Thus, if an object is to the right of the middle region, 
the user only hears sound in the right ear and vice versa for objects in the 
left region. When finding the destination, the user can position the destination 
in front and walk forward. If the user is actually off-center from the object in 
real life, the destination will eventually fall to the left or right as the user 
gets closer, which then forces the user to rotate and reposition in the 
middle. However, this strategy appeared limited since it only communicates 
three directions. Thus, the second strategy tried to improve upon this.

In the second iteration on regions, the audio system divided the field of view
into five regions in an attempt to get finer directional sound. This time, the
middle 10 degrees was devoted to the front. The 15 degree arcs to each side of
the middle indicates a partial left or right direction and the final 9 degrees 
represent a complete left or right. The front and complete left or right regions 
function in the same manner as the first design, but the 15 degree arcs are 
devoted to a partial representation of left or right. For instance, as the
location of an object sweeps to the right through the partial right region, the 
weight of the sound playing in the right ear slowly increases while the sound in
the left ear is slowly decreases. Compared to the first design, this strategy 
gives the user a better sense of slight changes in an object's direction. Though
it is easier with this design, the user must still listen carefully for the
shifts in stereo. But this is still better than the first version since the
complete left and right regions are there to give a clear indication of direction
should the user rotate too far away from the object. Aside from improving the
quality of directions, the authors also sought out pleasant sound cues.

Because audio is the primary means for users to interact with Guide Dog, it
needed a set of sound cues that would sound pleasant to the user and not become
annoying over time. The earliest implementation of the audio system used a
white noise tone as the beacon and sine wave tone to alert obstacles because
OpenAL could conveniently generate those tones. However, both proved to be far 
too abrasive and were discarded. The next attempts were great improvements.

In finding a good sound for the beacon, the authors wanted to emulate the sonar
sound from a submarine that one typically hears in movies. The second attempt
lead to a high pitched ping that sounded like hitting two metal pipes against
one another. It was serviceable, but as the pitch increased, it became more
unpleasant. The current version of the audio system uses a synthesized triple
beep sound that also conveys the sense of using sonar. It also has the
advantage of continuing to be pleasant sounding as the pitch increases. In
addition, it should be noted that the early white noise version of the beacon
denoted distance by increasing the volume and tempo as the user walked closer.
When the audio system started using the sonar sounds, increasing the tempo
while the sound was played became too unpleasant. Thus, pitch was chosen to
communicate distance.

For the second version of the obstacle sound, the authors wanted a sound that
could quickly grab the user's attention. For a while, the authors used a buzzing
sound, but after doing real user testing with the complete Guide Dog system,
the buzz was deemed to be too abrasive. This led to the use of a ding sound
similar to the one that is played when a person leaves a car door open. In
rapid succession, the ding is neutral enough that it does not become unpleasant
and it still sounds distinctive when played against the beacon.

Another design decision made in regards to obstacles was whether or not to alert
obstacles located to the complete left or right of the user. The user is told to
steer away from obstacles, but in practice, most obstacles that are alerted to the
left or right of the user do not obstruct his or her path if he or she continues
to walk forward. Thus, narrow walkways were often ignored. It was too confusing
to instruct the user to take heed of obstacle alerts to the side while avoid
the ones in front since the audio system used the same ding for both. Thus, the
authors tried turning off obstacles to the sides, which in practiced allowed
users to walk through narrow spaces. However, this is an area of the audio
system that needs further development.



\section{Discussion (\textasciitilde 1 column)}
\label{sec:discussion}

How well does your solution work, what are the next steps/what is the future
work, what are other applications of your technology? Any interesting insights
or lessons learned?

What you would you do differently or how would you re-design in the future?


\section{Conclusion}
\label{sec:conclusion}

In the domain of RGB-D camera software, the space of applications in which the
camera is mounted to the user remains relatively unexplored. Guide Dog attempts
to explore this space as a prototype guidance tool for users who are either
visually impaired or unable to see for other reasons. The Guide Dog system
described here makes many simplifying assumptions about its environment and
use cases, but research in related domains is very encouraging. Advances in
object detection and labeling could be incorporated to make Guide Dog's
destination guidance far more robust, and recent results in 3D mapping and
reconstruction could be incorporated to remove Guide Dog from the confines
of its local field of view. If incorporated into the system, these items
would make Guide Dog a useful tool for practical situations, and would bring
Guide Dog much closer to the initial vision of the authors. 

Guide Dog's implementation is publically available to use and build
upon~\cite{guidedog-website}.


\section{Acknowledgments}
Thank you to Dieter Fox and Kevin Lai for all of the helpful lab discussions and
ideas.

%
% The following two commands are all you need in the
% initial runs of your .tex file to
% produce the bibliography for the citations in your paper.
\bibliographystyle{abbrv}
\bibliography{guide-dog}  % guide-dog.bib is the name of the Bibliography in this case
% You must have a proper ".bib" file
%  and remember to run:
% latex bibtex latex latex
% to resolve all references
%
% ACM needs 'a single self-contained file'!
%
\balancecolumns
% That's all folks!
\end{document}
