\section{Introduction}
\label{sec:intro}

RGB-D cameras like the Microsoft Kinect~\cite{kinect-website} have recently
become popular because of their use in video game consoles. These cameras are
unique because not only to do they record color information, but they also record
depth information. This allows the gaming system to do a much better job in
figuring out what is happening in an environment. For example, it is much easier
to detect a person with and RGB-D camera than a color camera. With just a color
camera, complex and expensive computer vision techniques are required to detect
a person. However, this is simplified with an RGB-D camera because the person
will stand out from the background in the depth image.

Currently, the main use of RGB-D cameras is tracking people and using their
movements as inputs in a game. The RGB-D camera is generally placed facing the
user, in front of his or her TV. It captures the environment as the user moves
and passes this information to the video game system. The video game system then
processes this information. It generally extracts the location and movement of a
person in order to control the game. This is an exciting, but limited use of
this technology. Since the technology is young, not many other applications of
the technology have been explored.

Guide Dog explores a new application of RGB-D camera technology. Instead of
being used in a game, Guide Dog is a system that uses audio cues to direct a
user to a destination, while helping the user avoid obstacles. Guide Dog is a
unique application for an RGB-D camera since it is mounted on the user, rather
than facing the user, like general RGB-D camera applications. Guide Dog sees the
environment from the perspective of the user and is able to detect both the
destinations and any obstacles. Guide Dog communicates these locations to the
user by using two different audio tones: one for the destination and one for the
obstacles. The destination tone varies in pitch and direction. The pitch of the
tone indicates the distance from the destination and the direction indicates the
direction to the destination. The obstacle tone only starts when an obstacle is
within a few feet of the user, as only the close obstacles matter and too many
obstacles would overwhelm the user. The obstacle tone also uses direction to
indicate the direction of the obstacles.

This paper describes in detail the Guide Dog system as well as the authors'
experiences designing, building and improving Guide Dog. This paper describes
the various different approaches the authors experimented with when building
Guide Dog. It gives details about what worked, what didn't work, and why. It
gives details of how the individual components work and how they fit together
to form the Guide Dog system.

The paper is organized as follows. Section~\ref{sec:related} describes related
word. Section~\ref{sec:technical} describes the technical details of Guide Dog.
Section~\ref{sec:eval} evaluates the performance and success of Guide Dog.
Section~\ref{sec:discussion} discusses Guide Dog and
Section~\ref{sec:conclusion} concludes.
