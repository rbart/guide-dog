\section{Technical Details (\textasciitilde 2-3 pages)}
\label{sec:technical}

Guide Dog is composed of three separate components: destination detection,
obstacle detection and the audio interface. These components are described
below.

\begin{figure}
\includeimage{color.pdf}{8cm}{bb=0 0 474 474}
\caption{A sample scene, viewed in color. The destination object is the pink box
  in the middle. The two stacks of brown boxes are the left and right are
  obstacles.}
\label{fig:color}
\end{figure}

\subsection{Destination Detection}
\label{sec:dest}

\begin{figure}
\includeimage{destination.pdf}{8cm}{bb=0 0 474 474}
\caption{A similarity view showing regions of the scene with a high similarity
to the color of the destination object}
\label{fig:destination}
\end{figure}

\subsubsection{Overview}
\label{sec:dest-overview}

Guide Dog's destination detection component is responsible for identifying the
destination object in the scene, and determining it's position in space 
relative to the camera. The component is only able to detect objects of a solid
color that contrast highly from the scene. Functionality is provided for
recalibrating the system for a new color, or for different lighting or
exposure conditions. 

\subsubsection{Operation}
\label{sec:dest-op}

(Detailed description of the different functionalities and how they work. For
instance, describe your tracking / smoothing / recognition algorithm.)

\subsubsection{Implementation Details}
\label{sec:dest-impl}

(Some details on the software implementation.)

\subsection{Obstacle Detection}
\label{sec:obs}

\begin{figure}
\includeimage{obstacle.pdf}{5cm}{bb=0 0 300 300}
\caption{A view of the scene in Figure~\ref{fig:color}. This is a top down view.
  The user's location is shown with the blue X. The obstacles are shown labeled
  with red dots and the destination is labeled with a green dot.}
\label{fig:obstacle}
\end{figure}

\subsubsection{Overview}
\label{sec:obs-overview}

The second component of Guide Dog is the component to detect obstacles. This
component must detect the obstacles nearby to the user and convey the obstacles'
locations to the audio interface. Only obstacles near the user are detected in
order to prevent overwhelming the user. It is only important for the user to
know if he she is about to run into an obstacle, not if there is obstacle on the
other side of the room.

\subsubsection{Operation}
\label{sec:obs-op}

The obstacle detection algorithm looks only at the depth information from the
RGB-D camera. From this, it can see objects such as the floor, obstacles and
walls. Intuitvely, all of the obstacles will be above the floor, while the floor
itself is not an obstacle. This leads to a simple algorithm to detect the
obstacles: just remove the floor and everything left is an obstacle.

\subsubsection{Implementation Details}
\label{sec:pbs-impl}

The obstacle detection algorithm views the points in a format created by the
Point Cloud Library (PCL)\cite{pcl-website}. Each individual pixel viewed by
the camera is represented by an X coordinate, Y coordinate, depth and RGB color.
A data structure storing all of these points is called a ``point cloud'' and
represents the whole 3D environment the camera can see.

The obstacle detection algorithm first looks at the point cloud and extracts the
plane of the floor. PCL has a built-in planar segmentation library function that
detects the largest plane in view (there is an issue if the floor is not the
larget plane in view. This will be discussed in Section~\ref{sec:discussion}).
The plane detection algorithm gives the \emph{a}, \emph{b}, \emph{c}, and
\emph{d} coefficients of the in the following planar equation:

\begin{math}
ax + by + cz + d = 0
\end{math}

This results in an equation representing the plane of the floor. The obstacle
detection algorithm then removes all points that lie within a certain threshold
of the floor plane. The threshold helps remove noise added by the camera and is
currently set to 10 centimeters. After removing the floor, only the obstacles
are left. However, the obstacles are represented in 3D space, which is not
necessary for obstacle detection. This is because the height of an obstacle does
not matter. If there is an obstacle at any height, this needs to be
communicated to the user. This allows the obstacle detection algorithm to
convert the 3D obstacles in 2D space. To do this, the 3D obstacle coordinates
are projected onto the floor plane. Then, the floor plane is rotated so that it
is level. This provides a simple way to analyze the obstacles and calculate
distances.

TODO: where to talk about previous obstacle detection?

Once the obstacles have been extracted and projected into a 2D space more
analysis is performed to detect their locations. The obstacles are transferred
into a black and white image: a white pixel means there is part of an obstacle
at that location, a black pixel means that location is empty. An example of this
image is shown in Figure~\ref{fig:obstacle}. This image is then
passed off to a blob detection algorithm in OpenCV\cite{opencv-website}. The
blob detection algorithm groups together contiguous areas of points. This is
helpful because it takes a bunch of individual white pixels that are all next to
each other and groups them into one single ``blob'' with a single X and Y
coordinate location. As is shown in the example image in
Figure~\ref{fig:obstacle}, each obstacle is represented as a contiguous block
of white pixels. The blob detection algorithm detects the contiguous block and
produces a single point for the blob, marked with a red dot in the example
figure.

At this point, the obstacle detection algorithm has the locations of all
obstacles. There are two more steps that must happened before the obstacles are
sent off to the audio interface. First, the obstacle detection component has not
yet had any communication with the destination detection component. That means
that the obstacle detection component has no idea where the destinatio object is
and that the destination component will get detected and marked as an obstacle!
In order to prevent this, the obstacle detection component gets the coordinates
of the destination object from the destination detection component. It then
compares the coordinates of each obstacle it found with the coordinates of the
destination object. If any of the obstavle coordinates are close enough to the
destination, the obstacle detection component assumes that the obstacle must
actually be the desination and removes it from its list of obstacles. Second,
the obstacle detection component is only supposed to communicate obstacles that
are near to the user, not all obstacles. To do this, the obstacle detection
component simply removes any obstacles that are too far away from the user.

At this point, the remaining obstacles are ready to be passed off to the audio
interface so they can be communicated with the user.


\subsection{Audio Interface}
\label{sec:audio}

\begin{figure}
\includeimage{vsim.pdf}{6cm}{bb=0 0 371 371}
\caption{TODO: caption}
\label{fig:vsim}
\end{figure}

\subsubsection{Overview}
\label{sec:audio-overview}

(Provide the context of what your system does. A drawing might be useful here.)

\subsubsection{Operation}
\label{sec:audio-op}

(Detailed description of the different functionalities and how they work. For
instance, describe your tracking / smoothing / recognition algorithm.)

\subsubsection{Implementation Details}
\label{sec:audio-impl}

(Some details on the software implementation.)
