\section{Technical Details (\textasciitilde 2-3 pages)}
\label{sec:technical}

Guide Dog is composed of three separate components: destination detection,
obstacle detection and the audio interface. These components are described
below.

\begin{figure}
\includeimage{color.pdf}{8cm}{bb=0 0 474 474}
\caption{A sample scene, viewed in color. The destination object is the pink box
  in the middle. The two stacks of brown boxes are the left and right are
  obstacles.}
\label{fig:color}
\end{figure}

\subsection{Destination Detection}
\label{sec:dest}

\begin{figure}
\includeimage{destination.pdf}{8cm}{bb=0 0 474 474}
\caption{TODO: caption}
\label{fig:destination}
\end{figure}

\subsubsection{Overview}
\label{sec:dest-overview}

(Provide the context of what your system does. A drawing might be useful here.)

\subsubsection{Operation}
\label{sec:dest-op}

(Detailed description of the different functionalities and how they work. For
instance, describe your tracking / smoothing / recognition algorithm.)

\subsubsection{Implementation Details}
\label{sec:dest-impl}

(Some details on the software implementation.)

\subsection{Obstacle Detection}
\label{sec:obs}

\begin{figure}
\includeimage{obstacle.pdf}{5cm}{bb=0 0 300 300}
\caption{A view of the scene in Figure~\ref{fig:color}. This is a top down view.
  The user's location is shown with the blue X. The obstacles are shown labeled
  with red dots and the destination is labeled with a green dot.}
\label{fig:obstacle}
\end{figure}

\subsubsection{Overview}
\label{sec:obs-overview}

(Provide the context of what your system does. A drawing might be useful here.)

\subsubsection{Operation}
\label{sec:obs-op}

(Detailed description of the different functionalities and how they work. For
instance, describe your tracking / smoothing / recognition algorithm.)

\subsubsection{Implementation Details}
\label{sec:pbs-impl}

(Some details on the software implementation.)

\subsection{Audio Interface}
\label{sec:audio}

\begin{figure}
\includeimage{vsim.pdf}{6cm}{bb=0 0 371 371}
\caption{TODO: caption}
\label{fig:vsim}
\end{figure}

\subsubsection{Overview}
\label{sec:audio-overview}

(Provide the context of what your system does. A drawing might be useful here.)

\subsubsection{Operation}
\label{sec:audio-op}

(Detailed description of the different functionalities and how they work. For
instance, describe your tracking / smoothing / recognition algorithm.)

\subsubsection{Implementation Details}
\label{sec:audio-impl}

(Some details on the software implementation.)
